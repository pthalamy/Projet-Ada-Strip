\documentclass{article}

\usepackage[latin1]{inputenc} % un package
\usepackage[T1]{fontenc}      % un second package
\usepackage[francais]{babel}  % un troisi�me package
\usepackage{graphicx}
\usepackage{listings}
\usepackage[usenames, dvipsnames]{color}
\usepackage{framed}

\title{Rapport de Projet Algo -- Strip Packing}
\author{\textsc{Benjamin Lebit} - \textsc{Pierre Thalamy}}
\date{\today}

\begin{document}

\maketitle

\section {Mod�lisation des objets}
Le type \textit{Objet} repr�sente un rectangle � placer sur la bande. 
Afin de le mod�liser, nous avons cr�e un \textit{record type} contenant 
les attributs suivants:
\begin{itemize}
  \item Indice : Repr�sente l'indice de l'objet comme present dans le fichier 
    pars�.
  \item Largeur : Largeur du rectangle
  \item Hauteur : Hauteur du rectangle
  \item Position : De type \textit{Point}, aussi un \textit{record type} comprenant une 
    coordonn� X et une Y. 
    Repr�sente la position du coin sup�rieur gauche du rectangle dans le graph SVG.
\end{itemize}

Comme le type \textit{Objet} est declar� comme \textit{private}, nous avons cr�e tout un jeu de \textit{Setters} et \textit{Getters} pour lire ou modifier la valeur de ses attributs.
Finalement, \textit{Tableau\_Objets} est un tableau non-contraint d'�l�ments de type Objet.

\section {Parseur}
Le fonctionnement du \textit{Parseur} est simple, \textbf{Lecture\_En\_Tete} se contente de lire les deux premiers nombres de type \textit{Natural} qu'il trouve dans le fichier d'entr�. \\
La procedure \textbf{Lecture} quand � elle saute les deux premieres lignes du fichier, puis pour tous les objets du fichier, lit sa valeur d'indice, sa largeur, et sa hauteur, et les stocke dans les attributs de l'\textit{Objet} correspondant � l'int�rieur du tableau d'\textit{Objets}.
\paragraph{} Dans les deux cas, nous avons fait attention � bien g�rer toutes les exceptions qu'il est possible de rencontrer lors de la lecture du fichier � parser. Il reste cependant de la responsabilit� de l'utilisateur que de fournir un fichier d'entree valide, correctement format� et index�.
\newpage

\section {Mise en Oeuvre du packing}
Notre implementation de l'algorithme \textbf{Next\_Fit\_Decreasing\_Height} est la suivante :
\begin{itemize}
  \item \underline{Tri decroissant du tableau:} Nous avons choisi d'utiliser une procedure g�n�rique, \textit{Ada.Containers.Generic\_Array\_Sort}\footnote{http://rosettacode.org/wiki/Sort\_an\_array\_of\_composite\_structures} qui une fois instanci�e permet le tri d'un tableau par ordre croissant. Afin d'obtenir un tri d�croissant, nous avons remplace le signe '<' de la fonction de comparaison \textbf{"<"} par son oppos�, '>'.\\
\noindent Nous avons choisi cette solution car elle offre un tri rapide et concis.
  \item \underline{Positionnement des objets:} Chaque \textit{Objet} est donc dot� d'un attribut \emph{Position : Point(X,Y);} qui va lui �tre assign� par l'algorithme de \textit{Packing}. \textit{H\_Max} est la hauteur du premier �l�ment du niveau, \textit{H\_Cour} et \textit{L\_Cour} sont la hauteur et largeur courante permettant d'�tablir la position d'un objet dans l'image. \\
A la fin de l'algorithme, on obtient la hauteur totale du ruban, qui est �gale a la somme des hauteurs de tous les premiers �l�ments de niveau.
\end{itemize}

\section {Sauvegarde SVG}
Finalement, la procedure de sauvegarde dans le fichier SVG va ouvrir ou cr�er le fichier dont le nom est donn� en argument si celui-ci n'existe pas, puis fixer la taille de l'image SVG. Ensuite, pour tous les \textit{Objets} du tableau, il va y �crire le code de dessin d'un rectangle dot� des attributs suivants:

\begin{framed}
\small \noindent
<rect x="\textcolor{red}{Objet.Position.X}" y="\textcolor{red}{Objet.Position.X}" \\
width="\textcolor{red}{Objet.Largeur}" height="\textcolor{red}{Objet.Hauteur}" 
        style="\textcolor{red}{rgb(Vr,Vg,Vb)};\\
        stroke-width:0;stroke:rgb(0,0,0)"/>
\end{framed}

Les rectangles sont aussi color�s de fa�on cyclique afin qu'il n'y ai jamais deux rectangles c�te � c�te remplis de la m�me couleur.

\end{document}
